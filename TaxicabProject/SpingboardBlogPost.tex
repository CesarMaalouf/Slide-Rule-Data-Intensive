Because of smartphone apps like Uber, taxi politics has gotten a bunch of press lately. One of the claims that Uber proponents make is that normal taxis are inequitable. 

In New York City, taxis are heavily regulated. There are only about 14,000 taxis allowed to pick up passengers off the street in the center of the city. Uber gets around this regulation by only picking up passengers who requested a ride on the app.

I wanted to get some understanding about the users of taxis in New York City.  I looked at the locations where taxis drop off passengers. Then I looked to see if the locations correlate with demographic data about residents.

The drop off location data comes from the GPS in the taxi's meter. This data is recorded by the Taxi and Livery Cab Commission, which is nice enough to publish it on its website.

The demographic data comes from the United States Census Bureau.  They group NYC into about 2,000 regions that are usually just a few city blocks called census tracts. The data they gather includes race, age, commute mode, and income, among others.

Assigning a census tract to each GPS drop off point is no easy task. It's a computationally intensive job that I did

I found that the per capita income of residents is good at predicting the number of taxi drop offs. I did a Poisson regression and found that the number of drop offs per resident is proportional to the square of per capita income.

But it's important to keep in mind that drop offs do not necessarily represent residents. Something tells me that the 25 residents of Central Park didn't take 84,807 taxi rides each in 2013. There are many popular destinations within the city that attract people from everywhere. It seems like some of the same things make these locations attractive for the rich.

When I say that drop offs are proportional to the square of per capita income, this refers to the average that the model predicts. The model also expects that actual data will be randomly distributed around that average in a certain way. This is where a model based only on per capita income doesn't do so well. If per capita income were really the only thing that mattered, we would expect drop off data to be spread around the average by a certain amount. The actual data is spread about half a million times as much. Of course there is a lot more to the story than per capita income, but the large size of the extra spread implies that there is a lot more.










Because of smartphone apps like Uber, taxi politics has gotten a bunch of press lately. One of the claims that Uber proponents make is that normal taxis are inequitable. 

In New York City, taxis are heavily regulated. There are only about 14,000 taxis allowed to pick up passengers off the street in the center of the city. Uber gets around this regulation by only picking up passengers who requested a ride on the app.

I wanted to get some understanding about the users of taxis in New York City.  I looked at the locations where taxis drop off passengers. Then I looked to see if the locations correlate with demographic data about residents.

The drop off location data comes from the GPS in the taxi's meter. This data is recorded by the Taxi and Livery Cab Commission, which is nice enough to publish it on its website.

The demographic data comes from the United States Census Bureau.  They group NYC into about 2,000 regions that are usually just a few city blocks called census tracts. The data they gather includes race, age, commute mode, and income, among others.

Assigning a census tract to each GPS drop off point is no easy task. It's a computationally intensive job that I did using PostGIS. PostGIS has a has an efficient algorithm specifically for this job. But even with that, it would have taken too long for my laptop to match all 150 million drop offs.

Instead, I found the census tract of every point on a grid of every on one-thousandth of a degree. Then, I could find the tract of a drop off GPS point just by rounding to the nearest one thousandth of a degree. This reduced the work the computer needed to do by a factor of ten.

I found that the per capita income of residents is good at predicting the number of taxi drop offs. I did a Poisson regression and found that the number of drop offs per resident is proportional to the square of per capita income.

But it's important to keep in mind that drop offs do not necessarily represent residents. Something tells me that the 25 residents of Central Park didn't take 84,807 taxi rides each in 2013. There are many popular destinations within the city that attract people from everywhere. It seems like some of the same things make these locations attractive for the rich.

When I say that drop offs are proportional to the square of per capita income, this refers to the average that the model predicts. The model also expects that actual data will be randomly distributed around that average in a certain way. This is where a model based only on per capita income doesn't do so well. If per capita income were really the only thing that mattered, we would expect drop off data to be spread around the average by a certain amount. The actual data is spread about half a million times as much. Of course there is a lot more to the story than per capita income, but the large size of the extra spread implies that there is a whole lot more.

But using just income does do a pretty good job of explaining things. On a scale that goes from not explaining anything at all to explaining everything perfectly, using just income gets about 65% of the way to perfect.  For those who are familiar with regression, $R^2=0.65$.

Because of smartphone apps like Uber, taxi politics has gotten a bunch of press lately. One of the claims that Uber proponents make is that normal taxis are inequitable. 

In New York City, taxis are heavily regulated. There are only about 14,000 taxis allowed to pick up passengers off the street in the center of the city. Uber gets around this regulation by only picking up passengers who requested a ride on the app.

I wanted to get some understanding about the users of taxis in New York City.  I looked at the locations where taxis drop off passengers. Then I looked to see if the locations correlate with demographic data about residents.

The drop off location data comes from the GPS in the taxi's meter. This data is recorded by the Taxi and Livery Cab Commission, which is nice enough to publish it on its website.

The demographic data comes from the United States Census Bureau.  They group NYC into about 2,000 regions that are usually just a few city blocks called census tracts. The data they gather includes race, age, commute mode, and income, among others.

Assigning a census tract to each GPS drop off point is no easy task. It's a computationally intensive job that I did using PostGIS. PostGIS has a has an efficient algorithm just for this job. But even with that, it would have taken too long for my laptop to match all 150 million drop offs.

Instead, I found the census tract of every point on a grid of every on one-thousandth of a degree. Then, I could find the tract of a drop off GPS point just by rounding to the nearest one thousandth of a degree. This reduced the work the computer needed to do by a factor of ten.

I found that the per capita income of residents is good at predicting the number of taxi drop offs. I did a Poisson regression and found that the number of drop offs per resident is proportional to the square of per capita income.

But it's important to keep in mind that drop offs do not necessarily represent residents. Something tells me that the 25 residents of Central Park didn't take 84,807 taxi rides each in 2013. There are many popular destinations within the city that attract people from everywhere. It seems like some of the same things make these locations attractive for the rich.

When I say that drop offs are proportional to the square of per capita income, this refers to the average that the model predicts. The model also expects that actual data will be randomly distributed around that average in a certain way. This is where a model based only on per capita income doesn't do so well. If per capita income were really the only thing that mattered, we would expect drop off data to be spread around the average by a certain amount. The actual data is spread about half a million times as much. Of course there is a lot more to the story than per capita income, but the large size of the extra spread implies that there is a whole lot more.

It's possible that I could improve the model by adding more demographic factors, but my initial analysis says that it won't make a significant improvement. It could be that a non-demographic factor would make a big improvement. I would imagine that looking at the number of employees who work in a census tract would improve the model. 

Using just income does do a pretty good job of explaining things. On a scale that goes from not explaining anything at all to explaining everything perfectly, it gets about 65% of the way to perfect.  For those who are familiar with regression, $R^2=0.65$.



Because of smartphone apps like Uber, taxi politics has gotten a bunch of press lately. One of the claims that Uber proponents make is that normal taxis are inequitable.

In New York City, taxis are heavily regulated. There are only about 14,000 taxis allowed to pick up passengers off the street in the center of the city. Uber gets around this regulation by only picking up passengers who requested a ride on the app.

I wanted to get some understanding about the users of taxis in New York City.  I looked at the locations where taxis drop off passengers. Then I looked to see if the locations correlate with demographic data about residents.

The drop off location data comes from the GPS in the taxi's meter. This data is recorded by the Taxi and Livery Cab Commission, which is nice enough to publish it on its website.

The demographic data comes from the United States Census Bureau.  They group NYC into about 2,000 regions that are usually just a few city blocks called census tracts. The data they gather includes race, age, commute mode, and income, among others.

Assigning a census tract to each GPS drop off point is no easy task. It's a computationally intensive job that I did using PostGIS. PostGIS has a has an efficient algorithm just for this job. But even with that, it would have taken too long for my laptop to match all 150 million drop offs.

Instead, I found the census tract of every point on a grid of every on one-thousandth of a degree. Then, I could find the tract of a drop off GPS point just by rounding to the nearest one thousandth of a degree. This reduced the work the computer needed to do by a factor of ten.

A little note about the data: I removed Staten Island. For those not familiar with NYC geography, Staten Island is the island on the lower left of the map. It's much lighter than the rest of the city as it has few drop offs. It is economically, demographically, and culturally distinct from the rest of New York City. Even Staten Islanders agree: in 1993 they passed a referendum to secede from the city. The state of New York ignored them though. I respect the wishes of the residents and have seceded them from my analysis.

I found that the per capita income of residents is good at predicting the number of taxi drop offs. I did a Poisson regression and found that the number of drop offs per resident is proportional to the square of per capita income.

But it's important to keep in mind that drop offs do not necessarily represent residents. Something tells me that the 25 residents of Central Park didn't take 84,807 taxi rides each in 2013. There are many popular destinations within the city that attract people from everywhere. It seems like some of the same things make these locations attractive for the rich.

When I say that drop offs are proportional to the square of per capita income, this refers to the average that the model predicts. The model also expects that actual data will be randomly distributed around that average in a certain way. This is where a model based only on per capita income doesn't do so well. If per capita income were really the only thing that mattered, we would expect drop off data to be spread around the average by a certain amount. The actual data is spread about half a million times as much. Of course there is a lot more to the story than per capita income, but the large size of the extra spread implies that there is a whole lot more.

It's possible that I could improve the model by adding more demographic factors, but my initial analysis says that it won't make a significant improvement. It could be that a non-demographic factor would make a big improvement. I would imagine that looking at the number of employees who work in a census tract would improve the model.  It also could be that there are too many significant factors that are immeasurable. Popular locations like Central Park and Penn Station may attract people at significant rates. Directly measuring the number of people that travel to an area every day isn't possible.

Using just income does do a pretty good job of explaining things. On a scale that goes from not explaining anything at all to explaining everything perfectly, it gets about 65% of the way to perfect.  For those who are familiar with regression, $R^2=0.65$.

I, for one, find the simplicity of this model appealing: per capita drop offs are proportional to the square of per-capita income. While there are clearly more details, the simple take away here is quite satisfying.




The map below shows New York City, divided up by census tract. Darker census tracts have a high number of drop offs, while lighter ones have less. 



